\begin{frame}
\frametitle{An\'alisis de peor caso}
La cantidad m\'axima de True Positives ocurre cuando: \\
\vspace{3mm}
\begin{itemize}
\item \STPMP tiene la m\'axima cantidad de instancias con una X posibles. \\
\item \STPMN tiene la m\'axima cantidad de instancias con doble X posibles. \\
\end{itemize}
\vspace{3mm}
De esta forma todas las instancias con una X que sea posible considerar como True Positives del modelo van a serlo, evitando sus fallos en la clasificaci\'on.
\end{frame}



\begin{frame}
\frametitle{An\'alisis de peor caso}
Escribiendo formalmente la cota superior de aciertos para la clase
Positive: \\
\vspace{3mm}
\begin{itemize}
\item \STPMP = Min(\MTP, \IPTP - Min(\MFN, \INTP)) \\
% \vspace{3mm}
% \item $\STNMN$ = Min(\MTN, \INTN - Min(\MFP, \IPTN)) \\
\vspace{3mm}
\item \STPMN = Min(\MFN, \INTP) \\
% \vspace{3mm}
% \item $\STNFP$ = Min(\MFP, \IPTN) \\
\end{itemize}
\end{frame}



\begin{frame}
\frametitle{An\'alisis de peor caso}
Definiendo la cota superior a los aciertos del sistema, y superior a los aciertos, dado que se saben las confusiones del modelo: \\
\vspace{6mm}
$\STP$ $\leq$ Min(\MTP, \IPTP - Min(\MFN, \INTP)) + \\
\vspace{1mm}
\hspace{12.2mm}
Min(\MFN, \INTP) \\
\vspace{2mm}
$\SFN$ $\geq$ (\MTP - Min(\MTP, \IPTP - Min(\MFN, \INTP))) + \\
\vspace{1mm}
\hspace{12.2mm}
(\MFN - Min(\MFN, \INTP)) \\
\end{frame}