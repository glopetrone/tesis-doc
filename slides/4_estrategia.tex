\begin{frame}
\frametitle{Estrategia}
\vspace{3mm}
La estrategia del adversario entonces va a consistir en asignar a las instancias de tipo 3 prioritariamente como casos en los que el modelo falla. \\
\vspace{3mm}
Las instancias de tipo 1 y 4 son irrelevantes para el adversario por lo que le conviene asignarlas a aciertos del modelo.
\end{frame}



\begin{frame}[t]
\frametitle{Fallos en peor caso}
\vspace{6mm}
Para la clase Positive, la f\'ormula que da la cantidad de fallos totales basados en esa estrategia es la siguiente:
\vspace{3mm}
\begin{itemize}
    \item InstanciasTipo4 + Min(FallosDelModelo, InstanciasTipo3)
\end{itemize}
\end{frame}



\begin{frame}[t]
\frametitle{Fallos en peor caso}
\vspace{6mm}
Para la clase Positive, la f\'ormula que da la cantidad de fallos totales basados en esa estrategia es la siguiente:
\vspace{3mm}
\begin{itemize}
    \item InstanciasTipo4 + Min(\MFN, InstanciasTipo3)
\end{itemize}
\vspace{3mm}
$\symbol{92}$$\symbol{92}$ La cantidad de fallos del modelo es la confusion de Falsos Negativos
\end{frame}



\begin{frame}[t]
\frametitle{Fallos en peor caso}
\vspace{6mm}
Para la clase Positive, la f\'ormula que da la cantidad de fallos totales basados en esa estrategia es la siguiente:
\vspace{3mm}
\begin{itemize}
    \item InstanciasTipo4 + Min(\MFN, InstanciasTipo3)
\end{itemize}
\vspace{3mm}
\begin{itemize}
    \item El modelo falla para todas las instancias de tipo 3 que sea posible
\end{itemize}
\end{frame}



\begin{frame}[t]
\frametitle{Fallos en peor caso}
\vspace{6mm}
Para la clase Positive, la f\'ormula que da la cantidad de fallos totales basados en esa estrategia es la siguiente:
\vspace{3mm}
\begin{itemize}
    \item InstanciasTipo4 + Min(\MFN, InstanciasTipo3)
\end{itemize}
\vspace{3mm}
\begin{itemize}
    \item El modelo falla para todas las instancias de tipo 3 que sea posible
    \item Todas las instancias de tipo 4 suman un fallo
    \item Las instancias de tipo 1 no suman fallos
\end{itemize}
\end{frame}



\begin{frame}
\frametitle{Fallos en peor caso}
Si se pudiera saber la cantidad de instancias de cada tipo disponibles, se tiene la soluci\'on al peor caso. \\
\end{frame}



\begin{frame}[t]
\frametitle{Fallos en peor caso}
\vspace{6mm}
Instancias de tipo 1: \\
\vspace{3mm}
\begin{itemize}
\item Tanto la intervenci\'on Positive como Negative del sistema aciertan
\item Se saben las confusiones con intervenci\'on: \IPTP y \INTP
\item A su vez, en base a las assumptions, se sabe que las instancias que suman en \INTP tambi\'en est\'an en \IPTP ya que se est\'a agregando evidencia
\item \INTP es la cantidad de instancias tipo 1
\end{itemize}
\end{frame}



\begin{frame}[t]
\frametitle{Fallos en peor caso}
\vspace{6mm}
Instancias de tipo 4: \\
\vspace{3mm}
\begin{itemize}
\item Tanto la intervenci\'on Positive como Negative del sistema fallan
\item Se saben las confusiones con intervenci\'on: \IPFN y \INFN
\item A su vez, en base a las assumptions, se sabe que las instancias que suman en \IPFN tambi\'en est\'an en \INFN ya que se est\'a quitando evidencia
\item \IPFN es la cantidad de instancias tipo 4
\end{itemize}
\end{frame}



\begin{frame}[t]
\frametitle{Fallos en peor caso}
\vspace{6mm}
Instancias de tipo 3: \\
\vspace{3mm}
\begin{itemize}
\item Solo la intervenci\'on Positive del sistema acierta
\item Son las instancias restantes de la clase Positive
\end{itemize}
\end{frame}



\begin{frame}[t]
\frametitle{Fallos en peor caso}
\vspace{6mm}
La cantidad de instancias de tipo 3 es:
\vspace{3mm}
\begin{itemize}
\item CantidadPositive - \IPFN - \INTP
\item (\IPTP + \IPFN) - \IPFN - \INTP 
\item \IPTP - \INTP
\end{itemize}
\vspace{3mm}
Esta resta en la cuenta representa las instancias en las que el sistema solo acierta si el modelo acierta, coincidiendo con las de tipo 3.
\end{frame}



\begin{frame}[t]
\frametitle{Fallos en peor caso}
\vspace{6mm}
Recordando la f\'ormula de peor caso:
\vspace{3mm}
\begin{itemize}
    \item InstanciasTipo4 + Min(\MFN, InstanciasTipo3)
\end{itemize}
\vspace{3mm}
Reemplazando por las cantidades de instancias de cada tipo:
\vspace{3mm}
\begin{itemize}
    \item \IPFN + Min(\MFN, \IPTP - \INTP)
\end{itemize}
\end{frame}



\begin{frame}[t]
\frametitle{Fallos en peor caso}
\vspace{6mm}
Se puede hacer el mismo an\'alisis para la clase Negative, llegando a lo siguiente:
\vspace{3mm}
\begin{itemize}
    \item Clase Positive: \IPFN + Min(\MFN, \IPTP - \INTP)
    \item Clase Negative: \INFP + Min(\MFP, \INTN - \IPTN)
\end{itemize}
\vspace{3mm}
Resultando en el peor caso del sistema:
\vspace{3mm}
\begin{itemize}
    \item {\IPFN + Min(\MFN, \IPTP - \INTP) +

    \INFP + Min(\MFP, \INTN - \IPTN)}
\end{itemize}
\end{frame}