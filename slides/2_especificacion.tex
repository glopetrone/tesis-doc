\begin{frame}
\frametitle{Problema 1}
El primer problema a resolver entonces va a ser encontrar una funci\'on de costo para el modelo con las siguientes condiciones: \\
\vspace{3mm}
\begin{itemize}
\item La funci\'on devuelve el peor caso del sistema dado que fuera ejecutado con el modelo
\item Se tiene el resultado de ejecutar el sistema con intervenci\'on Positive y Negative en el modelo
\item Los datasets para evaluar el modelo y el sistema son el mismo
\end{itemize}
\end{frame}




\begin{frame}[t]
\frametitle{Notaci\'on}
\vspace{6mm}
Para definir el problema formalmente se introduce la notaci\'on $m^{f,D}_{ij}$ para denotar un valor de una matriz de confusi\'on: \\
\vspace{3mm}
$m^{f,D}_{ij}$ = $|$$\{d \in D $ $|$ $f(d_{1})=i \land d_{2}=j\}$$|$ \\
\vspace{3mm}
\begin{itemize}
\item $i$ es la clase real
\item $j$ es la clase que predice el modelo f sobre el dato
\item $D$ es el dataset utilizado, conteniendo un conjunto de pares $d_1$ instancia, $d_2$ clase real asociada, y su valor por defecto que representa todos los datos disponibles es $D$
\item $f$ es el modelo usado y puede valer:\\
\begin{itemize}
\item $M$ como el modelo a analizar\\
\item $S$ como el sistema completo con el modelo\\
\item $S_{*\rightarrow x}$ como el sistema con un modelo intervenido que predice de forma constante la clase $x$ para cualquier instancia\\
\end{itemize}
\end{itemize}
\end{frame}



\begin{frame}[t]
\frametitle{Especificaci\'on del problema}
\vspace{6mm}
Se asume lo siguiente sobre el comportamiento del sistema para cualquier conjunto de instancias $D$, modelo $M$ y sistema $S$: \\
\vspace{4mm}
\begin{minipage}{0.6\textwidth}
\begin{itemize}
\item \APmenos
\end{itemize}
\end{minipage} \\
\vspace{4mm}
$\symbol{92}$$\symbol{92}$ Agregar evidencia solo puede mejorar la predicci\'on
\end{frame}



\begin{frame}[t]
\frametitle{Especificaci\'on del problema}
\vspace{6mm}
Se asume lo siguiente sobre el comportamiento del sistema para cualquier conjunto de instancias $D$, modelo $M$ y sistema $S$: \\
\vspace{4mm}
\begin{minipage}{0.6\textwidth}
\begin{itemize}
\vspace{12.5mm}
\item \APmenos
\end{itemize}
\end{minipage} \\
\vspace{4mm}
$\symbol{92}$$\symbol{92}$ Restar evidencia solo puede empeorar la predicci\'on
\end{frame}



\begin{frame}[t]
\frametitle{Especificaci\'on del problema}
\vspace{6mm}
Se asume lo siguiente sobre el comportamiento del sistema para cualquier conjunto de instancias $D$, modelo $M$ y sistema $S$: \\
\vspace{4mm}
\begin{minipage}{0.6\textwidth}
\begin{itemize}
\item \APmas
\vspace{2mm}
\item \APmenos
\vspace{2mm}
\item \ANmas
\vspace{2mm}
\item \ANmenos
\end{itemize}
\end{minipage} \\
\vspace{4mm}
$\symbol{92}$$\symbol{92}$ Lo mismso para Clase Negative
\end{frame}



\begin{frame}
\frametitle{Especificaci\'on del problema}
Se espera lo siguiente sobre la funci\'on de costo asociada a las confusiones del sistema:\\
\vspace{4mm}
\begin{itemize}
\item \SCTP $\leq$ \SCFN \\
\vspace{2mm}
\item \SCTN $\leq$ \SCFP \\
\end{itemize}
\vspace{4mm}
Los fallos tienen mayor costo que los aciertos.\\
\vspace{4mm}
Para este trabajo se va a asumir que los aciertos tienen costo 1 y los fallos costo 0.
\end{frame}



\begin{frame}
\frametitle{Especificaci\'on del problema}
Datos disponibles:
\begin{table}[]
    \begin{minipage}{0.62\textwidth}
        \begin{tabular}{c|c|c|}
            & Pred Pos & Pred Neg \\ \hline
            Clase Pos & \IPTP & \IPFN \\ \hline
            Clase Neg & \IPFP & \IPTN \\ \hline
        \end{tabular}
    \end{minipage}
    \begin{minipage}{0.34\textwidth}
        \caption{Matriz de confusi\'on del sistema si Modelo = Positive}
    \end{minipage}
\end{table}
\begin{table}[]
    \begin{minipage}{0.62\textwidth}
        \begin{tabular}{c|c|c|}
            & Pred Pos & Pred Neg \\ \hline
            Clase Pos & \INTP & \INFN \\ \hline
            Clase Neg & \INFP & \INTN \\ \hline
        \end{tabular}
    \end{minipage}
    \begin{minipage}{0.34\textwidth}
        \caption{Matriz de confusi\'on del sistema si Modelo = Negative}
    \end{minipage}
\end{table}
\end{frame}



\begin{frame}
\frametitle{Especificaci\'on del problema}
Otros datos disponibles: \\
\vspace{3mm}
\begin{table}[]
    \begin{minipage}{0.60\textwidth}
        \begin{tabular}{c|c|c|}
            & Positive & Negative \\ \hline
            \#Instancias & P & N \\ \hline
        \end{tabular}
    \end{minipage}
    \begin{minipage}{0.38\textwidth}
        \caption{Cantidad de instancias por clase}
    \end{minipage}
\end{table}
\begin{table}[]
    \begin{minipage}{0.60\textwidth}
        \begin{tabular}{c|c|c|}
            & Pred Pos & Pred Neg \\ \hline
            Clase Pos & \MTP & \MFN \\ \hline
            Clase Neg & \MFP & \MTN \\ \hline
        \end{tabular}
    \end{minipage}
    \begin{minipage}{0.38\textwidth}
        \caption{Matriz de confusi\'on del modelo}
    \end{minipage}
\end{table}
\end{frame}



\begin{frame}
\frametitle{Especificaci\'on del problema}
Dados esos datos, se busca generar la peor matriz de confusi\'on posible del sistema que sea v\'alida (peor caso).
\vspace{3mm}
\begin{table}[]
    \begin{tabular}{c|c|c|}
        & Prediccion Positive & Prediccion Negative \\ \hline
        Clase Positive & \STP & \SFN \\ \hline
        Clase Negative & \SFP & \STN \\ \hline
    \end{tabular}
    \caption{Matriz de confusi\'on del sistema}
\end{table}
\end{frame}



\begin{frame}
\frametitle{Especificaci\'on del problema}
La peor es la que maximiza: \\
\vspace{3mm}
\begin{itemize}
\item {$\STP$ * $\SCTP$ + $\SFN$ * $\SCFN$ + 

$\SFP$ * $\SCFP$ + $\STN$ * $\SCTN$}
\end{itemize}
\vspace{3mm}
Como los costos asociados a las confusiones de aciertos valen 1 y las de los fallos 0, es equivalente a maximizar: \\
\vspace{3mm}
\begin{itemize}
\item $\SFN$ + $\SFP$
\end{itemize}
\vspace{3mm}
El an\'alisis tambi\'en vale para otros casos.
\end{frame}